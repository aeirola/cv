% LaTeX resume using res.cls
\documentclass[line,margin]{res} 
\usepackage{helvetica} % uses helvetica postscript font (download helvetica.sty)
%\usepackage{newcent}   % uses new century schoolbook postscript font 

% Language stuff for finnish
\usepackage[finnish]{babel}
%\usepackage[T1]{fontenc}
\usepackage[utf8]{inputenc}
\usepackage[pdftex,colorlinks,urlcolor=black]{hyperref}
\urlstyle{same}
\usepackage{datetime}

\pdfinfo{
 		/Author (Axel Eirola)
        /Title (Curriculum Vitae)
		/CreationDate (D:20080114150700)
		/ModDate (D:\pdfdate)
		/Subject (CV)
		/Keywords (Axel Eirola;CV)
}

% Expand left margin to make space for all long finnish words
\sectionwidth=1.5in

\begin{document}

\name{Axel Eirola}
% \address used twice to have two lines of address
\address{Servinkuja 5B52, 02150 Espoo. 040-7265959}
\address{\href{mailto:axel.eirola@gmail.com}{axel.eirola@gmail.com} \url{http://nonix.dy.fi/}}

 
\begin{resume}
 

\section{KOULUTUS} 
		{\sl Diploomi-Insinööritutkinnon opiskelija,} Teknillinen korkeakoulu, \hfill Syksy 2009-- \\
		pääaine ohjelmistojärjestelmät, sivuaine laskennallinen tiede ja tekniikka, 66/120 opinto\-pistettä.
		
		{\sl Tekniikan Kandidaatti,} Teknillinen korkeakoulu, \hfill Kesä 2009\\
		pääaine ohjelmistotekniikka, sivuaine laskennallinen tiede ja tekniikka.         	
		
		{\sl Ylioppilastutkinto,} Gymnasiet Lärkan 	\hfill Kevät 2005
                % \sl will be bold italic in New Century Schoolbook (or
	        % any postscript font) and just slanted in
		% Computer Modern (default) font
 

\section{TIETO\-TEKNIIKKA\-TAIDOT}
% \- denotates that the word may be hyphenated, if needed, at the specified position
		{\sl Ohjelmointikielet ja työkalut:} Python, C, Java, Scheme, Ruby, PHP, SQL, HTML, CSS, Apache, Eclipse, \LaTeX.  \\
		{\sl Käyttöjärjestelmät:} Windows, Mac OS X, Linux.

                
\section{KIELI\-TAIDOT}
		{\sl Ruotsi:} Äidinkieli \\
		{\sl Suomi:} Erinomainen \\
		{\sl Englanti:} Hyvä \\
		{\sl Saksa:} Alkeet

 
\section{TYÖKOKEMUS}
		{\sl Tutkimusapulainen,} TKK 	\hfill 2010 \\
			Aalto Ylopiston Teknillisen korkeakoulun Ohjelmistotekniikan laboratorion tutkimusapulaisena sekä opetustehtävissä.

		{\sl Tuntiopettaja,} TKK 	\hfill 2009 \\
			Teknillisen korkeakoulun Ohjelmistotekniikan laboratorion kursseilla T-106.3101 ``Ohjelmoinnin jatkokurssi T2 (C-kieli)'' sekä T-106.1150 ``Tietokone ja käyttöjärjestelmä'' tunti\-assistenttina.
		
		{\sl Tutkimusapulainen,} TKK 	\hfill Kesä 2008 \\
			Teknillisen korkeakoulun Tietojenkäsittelyteorian labaratorion tutkimusapulaisena 				ASP-tools projektin ohjelmistokehittäjänä.
				
		{\sl Tuntiopettaja,} TKK 	\hfill Syksy 2007 \\
			Teknillisen korkeakoulun Ohjelmistotekniikan laboratorion kurssin T-106.1002 
			``Datorn som verktyg'', sekä suomen\-kielisen vastineen T-106.1001 
			``Tieto\-kone työ\-välineenä'', tunti\-assistenttina.

		{\sl Järjestelmätuki, } YLE 	\hfill Kesä 2007 \\
			Määräaikaisessa kesätyössä Yleisradion uuden SAP R/3:en perustuvan HERA 
			Henkilöstö\-hallinto\-järjestelmän tuki- ja testaustehtävissä.
			
		{\sl ATK-opettaja ja -ylläpitäjä, } Everest Academy, Nepal \hfill Syksy 2005 \\
			Vapaaehtoistyönä neljä kuukautta paikallisessa englanninkielisessä koulussa Banepassa, Nepalissa.

		{\sl Avustava ATK-ylläpitäjä, } Gymnasiet Lärkan \hfill 2003--2005 \\
			Lukiossa avustavana ATK-ylläpitäjänä opiskeluaikana.

		{\sl Työharjoittelia, } Suomen Pankki	\hfill Syksy 2001 \\
			Kaksi viikkoa Suomen Pankin ATK-ylläpidossa osana yläasteen työharjoittelua.

\pagebreak
\section{AJOKORTTI}Luokka  B, syksystä 2005 lähtien.

 
\section{VARUSMIES\-PALVELUS}  
		RT ROHJ TJM (Rannikkotykistön rakettiohjusjoukkueen tulenjohtomies) \\ 
		Uudenmaan Prikaatissa 9.1 -- 7.7.2006, sotilasarvo korpraali.


\section{LUOTTAMUS\-TOIMET}             
            Tietokilta RY:n sisätoimikunnan aktiivijäsen. \hfill 2007--2009
            
\section{HARRASTUKSET}
            {\sl Harrastukset: } Tietotekniikka, elektroniikka, pyöräily, sulkapallo, purjehdus.

\end{resume}
\end{document}
